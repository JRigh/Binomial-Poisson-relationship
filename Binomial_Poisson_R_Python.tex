\documentclass[border=5mm, convert, usenames, dvipsnames,beamer]{standalone}
\usetheme{Madrid}
\usecolortheme{default}
%Information to be included in the title page:
\title{Sample title}
\author{Anonymous}
\institute{Overleaf}
\date{2021}

\usepackage[absolute,overlay]{textpos}

\defbeamertemplate*{frametitle}{}[1][]
{
    \begin{textblock*}{12cm}(1cm,0.75cm)
    {\color{purple} \fontsize{20}{43.2} \selectfont \insertframetitle}
    \end{textblock*}
    \begin{textblock*}{12cm}(1cm,2.5cm)
    {\color{purple} \fontsize{20}{24} \selectfont \insertframesubtitle}
    \end{textblock*}
}



\setbeamertemplate{footline}[frame number]
\usepackage{ragged2e}

\justifying
\usepackage{lmodern}
\usepackage{ImageMagick}
\usepackage[utf8] {inputenc}
\usefonttheme[onlymath]{serif}
\usepackage[english] {label}
\usepackage{amssymb}
\usepackage{amsmath}
\usepackage{bm}
\usepackage{bbm}
\usepackage[round] {natbib}
\usepackage{color}     
\usepackage{changepage}
\usepackage[export]{adjustbox}
\usepackage{graphicx}
\usepackage{minted}
\usepackage{listings}
\usepackage[svgnames]{xcolor}



\lstdefinestyle{R} {language=R,
    stringstyle=\color{DarkGreen},
    morekeywords={TRUE,FALSE},
    deletekeywords={data,frame,length,as,character},
    keywordstyle=\color{blue},
    commentstyle=\color{teal},    
    basicstyle=\ttfamily\tiny,
    breakatwhitespace=false,         
    breaklines=true,                 
    captionpos=b,                    
    keepspaces=true,                 
    numbers=left,                    
    numbersep=4pt,                  
    showspaces=false,                
    showstringspaces=false,
    showtabs=false,                  
    tabsize=2
}


\definecolor{codegreen}{rgb}{0,0.6,0}
\definecolor{codegray}{rgb}{0.5,0.5,0.5}
\definecolor{codepurple}{rgb}{0.58,0,0.82}

\lstdefinestyle{Python} {language=Python,
    commentstyle=\color{codegreen},
    keywordstyle=\color{magenta},
    numberstyle=\tiny\color{codegray},
    stringstyle=\color{codepurple},
    basicstyle=\ttfamily\tiny,
    breakatwhitespace=false,         
    breaklines=true,                 
    captionpos=b,                    
    keepspaces=true,                 
    numbers=left,                    
    numbersep=4pt,                  
    showspaces=false,                
    showstringspaces=false,
    showtabs=false,                  
    tabsize=2}


\makeatletter
\setbeamertemplate{frametitle}[default]{}
\makeatother
\usepackage{booktabs}
\newcommand{\ra}[1]{\renewcommand{\arraystretch}{#1}}


\begin{document}

\begin{frame}
\frametitle{Binomial and Poisson distributions: \\ rationale}

\vspace{55}
\noindent
The Binomial distribution is used to model the number of successes occuring in $n$ independent Bernoulli trials. The Poisson distribution is used to model a phenomenon that represents the count of events in a fixed time interval (or space unit). The r.v. $X$ represents then the number of successes in this interval.


\vspace{10}
\noindent
Suppose that $\lambda$ denotes the number of independent observations of an event over a given time interval. If we divide this interval into $n$ smaller intervals, then we find $ p = \frac{\lambda}{n} \Leftrightarrow \lambda = np$, the probability of observing a success over a time interval $n$ times smaller. Conditions for a Poisson approximation to the Binomial distribution:

\vspace{5}
$$
\begin{align*}
& (i) \ \ \ \ \ \  n  \text{  is large} \\
& (ii) \ \ \ \ \  n * p \text{  is small enough} \\
\end{align}
$$
\par
\end{frame}


\begin{frame}[ fragile]{}

\frametitle{Poisson distribution as an asymptotic\\
 case of the Binomial distribution}

\footnotesize
\vspace{55}
\noindent
$$
\begin{align*} 
 P(X=x) & = \binom{n}{x} p^{x} (1-p)^{n-x}\\
&  = \binom{n}{x} \bigg( \frac{\lambda}{n} \bigg)^{x}  \bigg(1-\frac{\lambda}{n} \bigg)^{n-x}\\
&  = \binom{n}{x} \frac{\lambda^{x}}{n^{x}} \bigg(1-\frac{\lambda}{n} \bigg)^{n}  \bigg(1-\frac{\lambda}{n} \bigg)^{-x}\\
& = \frac{n (n-1)...(n-x+1)}{x!} \frac{\lambda^{x}}{n^x} \bigg(1-\frac{\lambda}{n} \bigg)^{n}  \bigg(1-\frac{\lambda}{n} \bigg)^{-x}\\
&  = \frac{n (n-1)...(n-x+1)}{n^x} \frac{\lambda^{x}}{x!} \bigg(1-\frac{\lambda}{n} \bigg)^{n}  \bigg(1-\frac{\lambda}{n} \bigg)^{-x}\\
&  = \underbrace{1 \bigg( 1- \frac{1}{n} \bigg) \bigg( 1- \frac{2}{n} \bigg) ... \bigg( 1- \frac{x-1}{n} \bigg) }_{\approx 1} \frac{\lambda^{x}}{x!} \underbrace{\bigg(1-\frac{\lambda}{n} \bigg)^{n}}_{\approx e^{-\lambda}}  \bigg(1-\underbrace{\frac{\lambda}{n}}_{\approx 0} \bigg)^{-x}\\
& \approx  \frac{\lambda^{x}}{x!} e^{-\lambda} \ \ \ \ \ \ \  \ \ \ \text{as $n \to \infty$}
\end{align*}
$$

\end{frame}




\begin{frame}[ fragile]{}
\frametitle{Example of application}

\vspace{30}
\noindent
A machine tool produces $20,000$ pieces a day. The probability of one piece being defective is $0.1\%$. The probability that this machine produces exactly $20$ defective pieces is given by the Binomial distribution and equals $f_{X}(20)= P(X=20) = \binom{20000}{20} \ \ 0.001^{20} \ \ 0.999^{20 000-20} = 0.0889.$


\normalsize

\vspace{10}
\noindent
In the R programming language:
\begin{lstlisting}[style=
R]
dbinom(x = 20, size = 20000, prob = 0.001)
[1] 0.08887977
\end{lstlisting}

\vspace{5}
\noindent
In the Python language:
\begin{lstlisting}[style=Python]
from scipy.stats import binom
binom.pmf(20, n = 20000, p = 0.001)
0.08887976802110753
\end{lstlisting}

\par

\end{frame}

\begin{frame}[ fragile]{}

\frametitle{Poisson approximation}


\vspace{40}
\noindent
A machine tool produces $20,000$ pieces a day. The probability of one piece being defective is $0.1 \%$. The probability that this machine produces exactly $20$ defective pieces can be approximated by a Poisson distribution. If we let $\lambda = np$, that is, in our example $\lambda = 20000*0.001 = 20$, we can thus write $f_{X}(20)= P(X=20) = \frac{20^{20} e^{-20}}{20!} = 0.0888.$


\vspace{10}
\noindent
In the R programming language:
\begin{lstlisting}[style=R]
dpois(x = 20, lambda = 20000*0.001)
[1] 0.08883532
\end{lstlisting}

\vspace{5}
\noindent
In the Python language:
\begin{lstlisting}[style=Python]
from scipy.stats import poisson
poisson.pmf(20, mu = 20000*0.001)
0.0888353173920848
\end{lstlisting}

\end{frame}







\begin{frame}[ fragile]{}

\frametitle{References}

\vspace{30}
\noindent
Rizzo, M.L. (2019). Statistical Computing with R, Second Edition (2nd ed.). Chapman and Hall/CRC. \\
https://doi.org/10.1201/9780429192760

\vspace{10}
\noindent
The R Project for Statistical Computing:

\noindent
\urlf{https://www.r-project.org/}

\vspace{10}
\noindent
Python:

\noindent
\urlf{https://www.python.org/}


\vspace{10}
\noindent
course notes

\end{frame}

\end{document}
